% !TeX encoding = UTF-8
% !TeX root = SergeiKanaev-rus.tex
% !TeX spellcheck = ru_RU
\documentclass[11pt,a4paper,russian]{moderncv}

\moderncvtheme[grey]{classic}

\usepackage[T2A]{fontenc}
\usepackage[utf8]{inputenc}

\usepackage[scale=0.75]{geometry}

\name{Сергей}{Канаев}
\email{ksergy.91@gmail.com}                               % optional, remove / comment the line if not wanted
\social[github]{ksergy}                              % optional, remove / comment the line if not wanted
\extrainfo{skype: \texttt{fazedies}}                 % optional, remove / comment the line if not wanted

\makeatletter\renewcommand*{\bibliographyitemlabel}{\@biblabel{\arabic{enumiv}}}\makeatother

\begin{document}
\makecvtitle

\section{Образование}
\cventry{2008 --- 2013}{диплом магистра}{Севастопольский Национальный Университет Ядерной Энергии и Промышленности}{Институт Атомной Энергии}{Автоматизированное управление технологическими процессами (по отраслям)}{}

\section{Опыт работы}
\cventry{10.2016 --- 05.2018}{Инженер-программист}{Brogaming Studio / Starlab Studio}{Самара}{\newline{}Разработка сервера для мобильной игры}{За время работы выполнены задачи: %
\begin{itemize}
\item разработка и реализация тестового клиента для проверки функций сервера;
\item реализация алгоритма подбора соперников для партии;
\item разработка схемы БД, клент-серверного и межсерверного протокола для некоторых функций сервера;
\item реализация некоторых функций сервера;
\item повышение отказоустойчивости сервера;
\item оптимизация работы сервера;
\item составление bash/perl-скриптов для проведения нагрузочного тестирования.
\end{itemize}%
В работе использовались библиотеки \texttt{libpq}, \texttt{libsqlite}, \texttt{libev}, \texttt{jsoncpp} и \texttt{libconfig}. 
Также, была проведена формализация задачи для начинающегося проекта.%
}
\cventry{01.2015 --- 09.2016}{Инженер-программист}{Satellite Soft Labs}{Саратов}{\newline{}Разработка модулей SmartTrans NaviCore v3}{За время работы занимался разработкой: %
\begin{itemize}
\item серверов-приемников телематических данных от GPS/ГЛОНАСС-трекеров и ретрансляторов;
\item модулей обработки (фильтрования, выделения полезных сведений) полученных данных;
\item \texttt{bash}-скриптов для управления набором процессов как единым компонентом в терминах аналогичных демонам (службам) Linux;
\item модуля наблюдения за целостностью компонентов развернутой системы.
\end{itemize}%
Во время работы унифицировал интерфейс отдельных вычислительных модулей для использования их как динамически подгружаемых библиотек и уменьшения кол-ва вносимых правок кода. Реализовал собственные библиотеки, применямые в разработке продукта (C++ обертки над средствами ОС), для:%
\begin{itemize}
\item асинхронной работы с использованием механизма опроса \texttt{epoll} и \texttt{eventfd};
\item набора потоков (thread pool);
\item приема телематических сведений от трекеров по \texttt{TCP/IP};
\item кооперативной многозадачности (\texttt{coroutine});
\end{itemize} %
В той или иной степени использовал в работе:%
\begin{itemize}
\item библиотеки: \texttt{Boost.Asio}, \texttt{libmicrohttpd}, \texttt{cpp-netlib}, \texttt{libpqxx}, \texttt{libsqlite3}, \texttt{libdl};
\item подсистемы ядра Linux, встроенные средства: \texttt{bash}, \texttt{epoll}, \texttt{UNIX-сокеты};
\item многопоточность: конкурентную и кооперативную;
\item STL.
\end{itemize}}
\cventry{07.2013 --- 11.2014}{Инженер-программист}{Samsung R\&D Institute Ukraine}{Харьков}{\newline{}Системное программирование под ОС Tizen}{%
\begin{itemize}
\item Составил программу, использующую межпроцессное взаимодействие через DBus, управляющую процессами и сервисами с помощью API менеджера systemd.
\item Участововал в составлении прослоечной библиотеки, использующей библиотеку графического интерфейса EFL.
\item Участвовал в составлении промежуточной библиотеки – драйвера устройств ввода для xserver, вносил изменения в код модулей оконного менеджера enlightenment.
\item Составлял юнит-тесты и функциональные тесты с использованием библиотеки \texttt{check}.
\end{itemize}}

\section{Владение языками}
\cvitemwithcomment{Русский}{Родной}{}
\cvitemwithcomment{Английский}{Технический}{}

\section{Обо мне}
\par{Имею опыт в:}
\begin{itemize}
\item разработке на \texttt{C} и \texttt{C++} для ОС GNU/Linux CLI-приложений и GUI-приложений с использованием графических библиотек \texttt{EFL} и \texttt{QT}, включая межпроцессное взаимодействие посредством \texttt{UNIX}-сокетов, очередей (\texttt{FIFO}) и \texttt{DBus} (совсем немного);
\item составлении \texttt{bash}-скриптов для запуска последовательно включенных процессов (через \texttt{pipe}/\texttt{FIFO}) и управления как единым целым с возможностью остановки/перезапуска и конфигурирования отдельного набора процессов.
\end{itemize}
\par{Имею опыт в разработке через тестирование (TDD).}
\par{Во время разработки на \texttt{C++} использовал \texttt{Boost.Asio}, \texttt{STL}, кооперативную и конкурентную многопоточность.}
\par{Также имеется опыт разработки для микроконтроллеров \texttt{AVR}. Разрабатывал: систему управления комплексом шаговых двигателей для моделирования сопряженно-движущихся слоев жидкости; систему управления комплексом ШД для моделирования давлений воздуха в аэродинамической трубе. В обоих случаях разработотку вел вместе со схемотехнической частью.}
\par{Успешно участвовал в разработке системы наблюдения за транспортными средствами и их телеметрией, оповещения о событиях (слив, заправка, вход/выход в/из гео-зоны). Это требует разбора отдельных протоколов трекеров, фильтрования полученных сведений.}
\par{Имеется некоторый опыт работы с БД Postgre и SQLite 3.}
\par{Участвовал в разработке программы перевода растровой карты в мультигриды с подготовкой данных для алгоритмов выбора маршрута. Имеется свидетельство регистрации авторского права на продукт № 46694:}
\begin{itemize}
\item \textbf{Название:} Компьютерная программа ``Конвертация растровых картографических изображений в мультигриды с последующим преобразованием в структуры данных на основе графов''
\item \textbf{Авторы:} Михнев Сергей Сергеевич, Кулик Николай Сергеевич, Квасников Владимир Павлович, Канаев Сергей Валерьевич
\item \textbf{Имущественные права принадлежат:} Национальный авиационный университет, пр-т Космонавта Комарова, 1, г. Киев, 03680
\item \textbf{Дата регистрации:} 10.12.2012
\end{itemize}
\par{Есть интерес к разработке аналоговых и цифровых схем, сетевым протоколам.}

\clearpage
\end{document}
